% Options for packages loaded elsewhere
\PassOptionsToPackage{unicode}{hyperref}
\PassOptionsToPackage{hyphens}{url}
%
\documentclass[
  12pt,
  english,
  man]{apa6}
\usepackage{lmodern}
\usepackage{amssymb,amsmath}
\usepackage{ifxetex,ifluatex}
\ifnum 0\ifxetex 1\fi\ifluatex 1\fi=0 % if pdftex
  \usepackage[T1]{fontenc}
  \usepackage[utf8]{inputenc}
  \usepackage{textcomp} % provide euro and other symbols
\else % if luatex or xetex
  \usepackage{unicode-math}
  \defaultfontfeatures{Scale=MatchLowercase}
  \defaultfontfeatures[\rmfamily]{Ligatures=TeX,Scale=1}
\fi
% Use upquote if available, for straight quotes in verbatim environments
\IfFileExists{upquote.sty}{\usepackage{upquote}}{}
\IfFileExists{microtype.sty}{% use microtype if available
  \usepackage[]{microtype}
  \UseMicrotypeSet[protrusion]{basicmath} % disable protrusion for tt fonts
}{}
\makeatletter
\@ifundefined{KOMAClassName}{% if non-KOMA class
  \IfFileExists{parskip.sty}{%
    \usepackage{parskip}
  }{% else
    \setlength{\parindent}{0pt}
    \setlength{\parskip}{6pt plus 2pt minus 1pt}}
}{% if KOMA class
  \KOMAoptions{parskip=half}}
\makeatother
\usepackage{xcolor}
\IfFileExists{xurl.sty}{\usepackage{xurl}}{} % add URL line breaks if available
\IfFileExists{bookmark.sty}{\usepackage{bookmark}}{\usepackage{hyperref}}
\hypersetup{
  pdftitle={Key Drivers for Quality Coffee},
  pdfkeywords={Coffee},
  hidelinks,
  pdfcreator={LaTeX via pandoc}}
\urlstyle{same} % disable monospaced font for URLs
\usepackage{graphicx,grffile}
\makeatletter
\def\maxwidth{\ifdim\Gin@nat@width>\linewidth\linewidth\else\Gin@nat@width\fi}
\def\maxheight{\ifdim\Gin@nat@height>\textheight\textheight\else\Gin@nat@height\fi}
\makeatother
% Scale images if necessary, so that they will not overflow the page
% margins by default, and it is still possible to overwrite the defaults
% using explicit options in \includegraphics[width, height, ...]{}
\setkeys{Gin}{width=\maxwidth,height=\maxheight,keepaspectratio}
% Set default figure placement to htbp
\makeatletter
\def\fps@figure{htbp}
\makeatother
\setlength{\emergencystretch}{3em} % prevent overfull lines
\providecommand{\tightlist}{%
  \setlength{\itemsep}{0pt}\setlength{\parskip}{0pt}}
\setcounter{secnumdepth}{-\maxdimen} % remove section numbering
% Make \paragraph and \subparagraph free-standing
\ifx\paragraph\undefined\else
  \let\oldparagraph\paragraph
  \renewcommand{\paragraph}[1]{\oldparagraph{#1}\mbox{}}
\fi
\ifx\subparagraph\undefined\else
  \let\oldsubparagraph\subparagraph
  \renewcommand{\subparagraph}[1]{\oldsubparagraph{#1}\mbox{}}
\fi
% Manuscript styling
\usepackage{upgreek}
\captionsetup{font=singlespacing,justification=justified}

% Table formatting
\usepackage{longtable}
\usepackage{lscape}
% \usepackage[counterclockwise]{rotating}   % Landscape page setup for large tables
\usepackage{multirow}		% Table styling
\usepackage{tabularx}		% Control Column width
\usepackage[flushleft]{threeparttable}	% Allows for three part tables with a specified notes section
\usepackage{threeparttablex}            % Lets threeparttable work with longtable

% Create new environments so endfloat can handle them
% \newenvironment{ltable}
%   {\begin{landscape}\begin{center}\begin{threeparttable}}
%   {\end{threeparttable}\end{center}\end{landscape}}
\newenvironment{lltable}{\begin{landscape}\begin{center}\begin{ThreePartTable}}{\end{ThreePartTable}\end{center}\end{landscape}}

% Enables adjusting longtable caption width to table width
% Solution found at http://golatex.de/longtable-mit-caption-so-breit-wie-die-tabelle-t15767.html
\makeatletter
\newcommand\LastLTentrywidth{1em}
\newlength\longtablewidth
\setlength{\longtablewidth}{1in}
\newcommand{\getlongtablewidth}{\begingroup \ifcsname LT@\roman{LT@tables}\endcsname \global\longtablewidth=0pt \renewcommand{\LT@entry}[2]{\global\advance\longtablewidth by ##2\relax\gdef\LastLTentrywidth{##2}}\@nameuse{LT@\roman{LT@tables}} \fi \endgroup}

% \setlength{\parindent}{0.5in}
% \setlength{\parskip}{0pt plus 0pt minus 0pt}

% \usepackage{etoolbox}
\makeatletter
\patchcmd{\HyOrg@maketitle}
  {\section{\normalfont\normalsize\abstractname}}
  {\section*{\normalfont\normalsize\abstractname}}
  {}{\typeout{Failed to patch abstract.}}
\makeatother
\shorttitle{A Multi-Variate Analysis}
\author{Bosniack, Lesley, Bowen, Janet, Higgins, Thomas, \& Palacios, Gerardo}
\affiliation{
\vspace{0.5cm}
\textsuperscript{1} DSC 424 - Advanced Data Analysis\\\textsuperscript{2} DePaul University}
\keywords{Coffee\newline\indent Word count: X}
\DeclareDelayedFloatFlavor{ThreePartTable}{table}
\DeclareDelayedFloatFlavor{lltable}{table}
\DeclareDelayedFloatFlavor*{longtable}{table}
\makeatletter
\renewcommand{\efloat@iwrite}[1]{\immediate\expandafter\protected@write\csname efloat@post#1\endcsname{}}
\makeatother
\usepackage{csquotes}
\usepackage{endfloat}
\usepackage{setspace}\doublespacing
\usepackage{lineno}
\usepackage{lscape}
\usepackage{lipsum}
\usepackage{float}
\usepackage{setspace}
\usepackage{booktabs}
\usepackage{graphicx}
\usepackage{multicol}
\usepackage[backend=biber]{biblatex}
\addbibresource{r-references.bib}
\usepackage{afterpage}
\newcommand{\blandscape}{\begin{landscape}}
\newcommand{\elandscape}{\end{landscape}}
\ifxetex
  % Load polyglossia as late as possible: uses bidi with RTL langages (e.g. Hebrew, Arabic)
  \usepackage{polyglossia}
  \setmainlanguage[]{english}
\else
  \usepackage[shorthands=off,main=english]{babel}
\fi
\usepackage[]{biblatex}
\addbibresource{r-references.bib}

\title{Key Drivers for Quality Coffee}

\date{}

\begin{document}
\maketitle

\singlespacing

\hypertarget{domain-knowledge-and-data-gathering}{%
\section{Domain Knowledge and Data Gathering}\label{domain-knowledge-and-data-gathering}}

According to the Coffee Quality Institute (CQI), coffee quality is one of the most important variables that influence a coffee's value. One way that coffee quality can be measured is through a blind tasting, also known as cupping, by certified coffee analysts using the SCAA Cupping Protocol. This protocol gives guidelines for evaluation ranging from necessary equipment to preparation of the coffee. Ratings are given on a scale of 6-point scale in various categories such as aroma, flavor, aftertaste, acidity, body, balance, sweetness, clarity, consistency, and overall impression. The final grade sums the ratings against a total of 100 points, similar to the scale that exists for wine. Anything rated over 80 points is considered a premium coffee. This method of evaluation provides a consistent and objective methodology for capturing some of the beans' sensory aspects and for evaluating quality.
~

The CQI's goal is to improve the quality of coffee and the lives of coffee producers. As a result, the CQI has provided opportunities to individuals to become certified as coffee graders, analysts among other experts to support and educate current and future generations of coffee growers. The data-set selected and described below was compiled by the CQI from samples submitted for evaluation. It provides a profile of coffee growers and coffee beans and the quality of the coffee grown, as measured according to the SCAA categories. Analysis of this data-set could identify valuable findings that would strengthen growing practices and knowledge which may favorably affect coffee producers
. The final data-set was gathered from multiple sources and combined into a single data-set. The parent data was gathered from \emph{\url{https://www.kaggle.com/volpatto/coffee-quality-database-from-cqi}} representing 1339 observations and 43 variables from the CQI database of coffee ratings from 2010 through 2018. Each observation represents a coffee grower and 43 variables. The relevant weather and geographical data associated with the growing regions of the parent data-set was gathered using two APIs to capture their potential impact on coffee quality. Isolating this data for the coffee growing season of each region is necessary for the weather data to be useful. Because growing season was not easily identified, data on harvest season was collected by country and if available, by region from a number of sources. Harvest time frame will be utilized to back into growing season.

\hypertarget{api-methodologies}{%
\subsection{API Methodologies}\label{api-methodologies}}

\hypertarget{google-place-api}{%
\subsubsection{Google Place API}\label{google-place-api}}

The Google Places API is a service used to gather data from any given location. The available request, \enquote{Place Search} , was used to determine the Latitude and Longitude for each of the given countries regions and countries. A custom user function was created using Excel VBA to build an http request based on the name of the given region in the Kaggle data-set. The GET request returns a JSON file which was then parsed to focus on the central coordinates of any given region in the data-set.

\hypertarget{darkskies-api}{%
\subsubsection{DarkSkies API}\label{darkskies-api}}

The Dark sky API is a comprehensive weather API that allows users to look up weather data from anywhere in the world. The API is very simple to use and only requires 2 or 3 parameters (Lat\textbar Long and/or Time). The weather data was gathered based on the estimated harvest season months for each of the regions. The return is beginning of month weather measurements for that particular region and time. The readings associated then is calculated to determine an average measure for the harvest season.

\hypertarget{data-pre-processing}{%
\subsection{Data Pre-processing}\label{data-pre-processing}}

The data gathered from the API's came in a clean format which did not need further processing. The parent data-set, however, required the following steps in order to be correctly processed by R.

\begin{itemize}
\item
  The numerical variable \enquote{Weight}, came in a \enquote{String} format containing mixed unit and measurements for coffee (kg or lbs).The variable was then separated so that all numerical units were separated and converted under a unified unit (kg).
\item
  The numerical variable \enquote{Harvest Year}, came in a \enquote{String} format containing mixed start \textbar{} end harvest years. To standardize the column, only the end Harvest year was taken.
\item
  The numerical variable \enquote{Grading Date} and \enquote{Expiration} came in a \enquote{String} format containing the date that the coffee was graded or expired. In order for R to correctly process that variable it needed to be converted into a time variable. The year, day and month for each row was extracted to build the correct date value in order for R to interpret.
\item
  Data needed to be reduced to represent only complete data.
\item
  Values of each variable were reviewed. Data adjustments were identified for potentially erroneous values.
\end{itemize}

\hypertarget{variable-survey}{%
\subsection{Variable Survey}\label{variable-survey}}

The main purpose with this data-set is to find the \emph{key drivers} for well rated coffee or quality coffee. The predictors generally fall under 5 categories that portray different characteristics for each of the coffee producers.

\begin{tabular}{>{\raggedright\arraybackslash}p{1in}>{\raggedright\arraybackslash}p{5in}}
\toprule
Category & Description\\
\midrule
Grader scores & This is a set of scores given to a sample by a certified Q grader, each corresponding with different aspects of quality in the coffee. Cupper.Points is a subjective quality evaluation made by the grader, and Total.Cup.Points is the sum total of all of these. These are all numeric variables, and most of them are normally distributed, with the exception of Uniformity, Clean.Cup, and Sweetness, which are strongly skewed left - most values are at 10 for those.\\
Geographical data & These are traits of the coffee beans noted during inspection; Color is a categorical variable, the others are numeric (although treating them as categorical might be an option). Category One defects are considered more undesirable than Category Two defects. For color, most of the data falls into two categories; for the others, the data is skewed right, with most values at zero (in other words, no defects of that sort were observed).\\
Production characteristics & Data associated with where the graded coffee sample was produced; used to get weather information, or possibly as a variable of its own, depending on analysis methodology. Categorical variables; some countries/regions are represented much more than others.\\
Weather data & Variables that are in some way associated with the production process of the coffee. Mix of categorical and numerical variables.\\
Other quality indicators & Variables describing the weather conditions in the region the coffee was produced. Mostly numeric; the categorical variables might not be usable. In general, the idea here is that growing conditions are known to affect the quality of farm products\\
\bottomrule
\end{tabular}

~

\hypertarget{notable-distributions-and-transformations}{%
\subsubsection{Notable Distributions and Transformations}\label{notable-distributions-and-transformations}}

The majority of the variables were normally distributed with the exception of a few that may need special consideration when running regressions or conducting PCA analysis. 6 variables were shown to have non-normal distributions.

\begin{enumerate}
\def\labelenumi{\arabic{enumi}.}
\item
  Number of Bags
  \includegraphics{CoffeePaper_files/figure-latex/Bags-1.pdf}
  The number of bags has a highly skewed-right bi-modal distribution at zero and 250 bags. A square root transformation is shown to improve the normality of the distribution.
\item
  Weight and Num of Bags x Weight
  \includegraphics{CoffeePaper_files/figure-latex/Weight-1.pdf}
  Both variables are highly skewed-right with most of the observations generally are generally light. A logarithmic transformation is shown to have the most improvement.
\item
  Species
  \includegraphics{CoffeePaper_files/figure-latex/Species-1.pdf}
  Species is highly disproportionate with most of the observations falling under \emph{Arabica}. Only 33 observations from the final data-set are categorized as \emph{Robusta}.
\item
  Color
\end{enumerate}

\begin{verbatim}
## [[1]]
\end{verbatim}

\includegraphics{CoffeePaper_files/figure-latex/Color-1.pdf}
The color of the coffee largely falls into a single category making it a highly uneven distribution between the rest of the categories. 638 observations fell under \enquote{Green} while the remaining 167 observations were spread evenly between 3 categories.

\begin{enumerate}
\def\labelenumi{\arabic{enumi}.}
\setcounter{enumi}{4}
\item
  Category One and Two Defects
  \includegraphics{CoffeePaper_files/figure-latex/Category-1.pdf}
  Category one and two defects are strongly skewed to the right with the majority of the values at or near 0. This distribution is anticipated as we would expect most coffees would be devoid of any defects. Both the logarithmic and square root transformations improve the distributions by reducing skew but the improvement is only nominal.
\item
  Altitudes (Low, High and Mean)
  \includegraphics{CoffeePaper_files/figure-latex/Altitudes-1.pdf}
  All three altitude variables were strongly skewed right. A logarithmic transformation was shown to have the biggest improvement on the distribution. It is worth noting that the most of the \enquote{High} and \enquote{Low} values are identical. With the \enquote{Mean} variable depicting the most variance between the three. Going forward the mean will be the sole variable to use relating to the altitude.
  ~
\end{enumerate}

\hypertarget{notable-correlations}{%
\subsubsection{Notable Correlations}\label{notable-correlations}}

Correlations were reviewed across all variables. Below are initial plots against all variables as well as whose correlations of above 0.65 or below -0.65. All of these variables exhibit a positive correlations. It can be noted that altitude variables all show perfect multi-collinearity suggesting repetitive data. This supports our earlier suggestion on the altitude variables. The Specialty Coffee Association of America (\enquote{SCAA}) quality scores are all positively correlated with the lowest correlation of 0.51 between aroma and body. This suggests that if quality is high on some criteria, it is likely that quality will be high for other criteria and for the overall quality score, Total.Cup.Points. Similarly, where low quality scores exist on some criteria, one would expect low quality scores on other criteria and on the combined score. Defects 1 and 2 and moisture levels exhibit mild negative correlations to the SCAA scores. Altitude does not appear to be correlated with SCAA scores.

\includegraphics{CoffeePaper_files/figure-latex/coffee corr-1.pdf}

The weather data-set consist of three categories of measurements. Hourly, daily and currently. The most complete section of the data consisted of the \enquote{daily} measurements This consisted of 39 additional variables related to the geographical location of any given coffee producer. Nearly all of the variables are correlated with each other while a few are negatively correlated. The variables humidity, pressure and visibility all tend to decrease in value while all other variables increase. The correlation groupings of at least 0.75 display a similar pattern. The strongest group of correlated variables are all positive with only one,(pressure) that is negative. From the graph, you can see two groupings of strongly correlated variables, the moon phases and the daily temperatures, and the other daily apparent temperatures and daily pressure.

\includegraphics{CoffeePaper_files/figure-latex/weather corr-1.pdf}

\hypertarget{tables}{%
\section{Tables}\label{tables}}

\afterpage{
\begin{landscape}
\begin{center}

\begin{tabular}{>{\raggedright\arraybackslash}p{1.5in}>{\raggedright\arraybackslash}p{1in}>{\raggedright\arraybackslash}p{3.1in}>{\raggedright\arraybackslash}p{3.1in}}
\toprule
Variable & Type & Description & Relevance\\
\midrule
Species & Categorical &  & Different species of coffee cherry may have different qualities that impact the score\\
Number.of.Bags & Numeric & Number of Bags of Coffee in a Growing Season & Small batches are often more premium\\
Wt & Numeric & Average weight of bags in kilograms & Small batches are often more premium\\
Harvest.Year.Fixed & Date & Year the coffee was harvested & For similar products (like wine), some years are known to be good/bad for quality historically\\
Grading.Date.Fixed & Date & Month, day and year coffee was graded & Probably not relevant\\
\addlinespace
Variety & Categorical & Variety of coffee plant & Different varieties sometimes have unique characteristics\\
Processing.Method & Categorical & Method of processing (wet/washed, dry, pulped and semi-washed) & Processing method can alter taste\\
Aroma & Numeric & SCAA aroma (dry fragrance) quality score ranging from 6-10 & Higher values have a better aroma\\
Flavor & Numeric & SCAA flavor quality score ranging from 6-10 & Higher values have a better flavor\\
Aftertaste & Numeric & SCAA aftertaste quality score ranging from 6-10 & Higher values have a better aftertaste\\
\addlinespace
Acidity & Numeric & SCAA acidity quality score ranging from 6-10 & Higher values have a better flavor profile\\
Body & Numeric & SCAA body quality score ranging from 6-10 & Higher values have better body\\
Balance & Numeric & SCAA balance quality score ranging from 6-10 & Higher values have better balance\\
Uniformity & Numeric & SCAA uniformity quality score ranging from 6-10 & Higher values have better unifomity\\
Clean.Cup & Numeric & SCAA clean cup quality score ranging from 6-10 & Higher values have better clen up\\
\bottomrule
\end{tabular}


\begin{tabular}{>{\raggedright\arraybackslash}p{1.5in}>{\raggedright\arraybackslash}p{1in}>{\raggedright\arraybackslash}p{3.1in}>{\raggedright\arraybackslash}p{3.1in}}
\toprule
Variable & Type & Description & Relevance\\
\midrule
Sweetness & Numeric & SCAA sweetness quality score ranging from 6-10 & Higher values have better sweetness\\
Cupper.Points & Numeric & Correction points added by grader for more appealing coffees not reflected in other scores & Grader's personal evaluation\\
Total.Cup.Points & Numeric & Sum of Aroma, Flavor, Aftertaste, Acidity, Body, Balance, Uniformity, Clean Cup and Sweetness scores & Total score\\
Moisture & Numeric & Moisture percentage & Important to roasters for quality control.  Contributes to acidity and aroma.\\
Category.One.Defects & Numeric & Number of primary defects & Catch-all for detrimental bean traits noticed by grader\\
\addlinespace
Quakers & Numeric & Quantity of unripe or poorly roasted beans & Unripe beans, sign of bad nutrition/picking practices/etc\\
Color & Categorical & Coffee color ranges from green to blue & Factors during cultivation, picking, drying, and milling can impact bean color and give clues about the characteristics a particular bean will take on during roasting and brewing\\
Category.Two.Defects & Numeric & Number of secondary defect & Less severe detrimental bean traits\\
Expiration.Fixed & Date & One year post grade date & Probably not relevant\\
altitude\_low\_meters & Numeric & Distance from Sea Level - minimum for grower & \\
\addlinespace
altitude\_high\_meters & Numeric & Distance from Sea Level - maximum for grower & Higher altitudes are often associated with better quality coffee\\
altitude\_mean\_meters & Numeric & Distance from Sea Level - average for grower & Mean altitude range of coffee farmers\\
Regional Longitude & N/A & Longitude coordinate of coffee producer & Provides primary key to merge weather data\\
Regional Latitude & N/A & Latitude coordinate of coffee producer & Provides primary key to merge weather data\\
Apparent (feels-like) temperature & Numeric & How weather conditions feel to bare skin & Weather patterns can affect crop production and quality\\
\bottomrule
\end{tabular}


\begin{tabular}{>{\raggedright\arraybackslash}p{1.5in}>{\raggedright\arraybackslash}p{1in}>{\raggedright\arraybackslash}p{3.1in}>{\raggedright\arraybackslash}p{3.1in}}
\toprule
Variable & Type & Description & Relevance\\
\midrule
Atmospheric pressure & Numeric & Pressure due to weight of atmosphere & Weather patterns can affect crop production and quality\\
Cloud cover &  &  & Weather patterns can affect crop production and quality\\
Dew point & Numeric & Atmospheric temperature at which dew will form & Weather patterns can affect crop production and quality\\
Humidity & Numeric & Concentration of water vapor in air & Weather patterns can affect crop production and quality\\
Liquid precipitation rate & Numeric &  & Weather patterns can affect crop production and quality\\
\addlinespace
Moon phase &  &  & Gravitational force of the moon can stimulate plant growth and crop quality\\
Nearest storm distance & Numeric &  & Weather patterns can affect crop production and quality\\
Nearest storm direction & Numeric &  & Weather patterns can affect crop production and quality\\
Ozone &  &  & Ozone can damage plants and reduce their survival\\
Precipitation type &  &  & Weather patterns can affect crop production and quality\\
\addlinespace
Snowfall &  &  & Weather patterns can affect crop production and quality\\
Sun rise/set &  &  & Weather patterns can affect crop production and quality\\
Temperature &  &  & Weather patterns can affect crop production and quality\\
Text summaries & N/A &  & Not relevant - text blurb about daily weather\\
UV index &  &  & Weather patterns can affect crop production and quality\\
\bottomrule
\end{tabular}


\begin{tabular}{>{\raggedright\arraybackslash}p{1.3in}>{\raggedright\arraybackslash}p{1in}>{\raggedright\arraybackslash}p{3.1in}>{\raggedright\arraybackslash}p{3.1in}}
\toprule
Variable & Type & Description & Relevance\\
\midrule
Atmospheric pressure & Numeric & Pressure due to weight of atmosphere & Weather patterns can affect crop production and quality\\
Cloud cover &  &  & Weather patterns can affect crop production and quality\\
Dew point & Numeric & Atmospheric temperature at which dew will form & Weather patterns can affect crop production and quality\\
Humidity & Numeric & Concentration of water vapor in air & Weather patterns can affect crop production and quality\\
Liquid precipitation rate & Numeric &  & Weather patterns can affect crop production and quality\\
\addlinespace
Moon phase &  &  & Gravitational force of the moon can stimulate plant growth and crop quality\\
Nearest storm distance & Numeric &  & Weather patterns can affect crop production and quality\\
Nearest storm direction & Numeric &  & Weather patterns can affect crop production and quality\\
Ozone &  &  & Ozone can damage plants and reduce their survival\\
Precipitation type &  &  & Weather patterns can affect crop production and quality\\
\addlinespace
Snowfall &  &  & Weather patterns can affect crop production and quality\\
Sun rise/set &  &  & Weather patterns can affect crop production and quality\\
Temperature &  &  & Weather patterns can affect crop production and quality\\
Text summaries & N/A &  & Not relevant - text blurb about daily weather\\
UV index &  &  & Weather patterns can affect crop production and quality\\
\bottomrule
\end{tabular}
\end{center}
\end{landscape}}

\afterpage{
\begin{landscape}
\begin{center}

\begin{tabular}{llllllllll}
\toprule
\multicolumn{1}{c}{\begingroup\fontsize{12}{14}\selectfont Statistic\endgroup} & \multicolumn{1}{c}{\begingroup\fontsize{12}{14}\selectfont Latitude\endgroup} & \multicolumn{1}{c}{\begingroup\fontsize{12}{14}\selectfont Longitude\endgroup} & \multicolumn{1}{c}{\begingroup\fontsize{12}{14}\selectfont Number.of.Bags\endgroup} & \multicolumn{1}{c}{\begingroup\fontsize{12}{14}\selectfont Weight\endgroup} & \multicolumn{1}{c}{\begingroup\fontsize{12}{14}\selectfont Harvest.Year\endgroup} & \multicolumn{1}{c}{\begingroup\fontsize{12}{14}\selectfont Harvest.Length\endgroup} & \multicolumn{1}{c}{\begingroup\fontsize{12}{14}\selectfont Aroma\endgroup} & \multicolumn{1}{c}{\begingroup\fontsize{12}{14}\selectfont Flavor\endgroup} & \multicolumn{1}{c}{\begingroup\fontsize{12}{14}\selectfont Aftertaste\endgroup}\\
\midrule
\multicolumn{1}{c}{\begingroup\fontsize{12}{14}\selectfont 1st Qu.\endgroup} & \multicolumn{1}{c}{\begingroup\fontsize{12}{14}\selectfont 2.536\endgroup} & \multicolumn{1}{c}{\begingroup\fontsize{12}{14}\selectfont -91.47\endgroup} & \multicolumn{1}{c}{\begingroup\fontsize{12}{14}\selectfont 20.0\endgroup} & \multicolumn{1}{c}{\begingroup\fontsize{12}{14}\selectfont 1.0000\endgroup} & \multicolumn{1}{c}{\begingroup\fontsize{12}{14}\selectfont 2012\endgroup} & \multicolumn{1}{c}{\begingroup\fontsize{12}{14}\selectfont 92.0\endgroup} & \multicolumn{1}{c}{\begingroup\fontsize{12}{14}\selectfont 7.420\endgroup} & \multicolumn{1}{c}{\begingroup\fontsize{12}{14}\selectfont 7.330\endgroup} & \multicolumn{1}{c}{\begingroup\fontsize{12}{14}\selectfont 7.170\endgroup}\\
\multicolumn{1}{c}{\begingroup\fontsize{12}{14}\selectfont 3rd Qu.\endgroup} & \multicolumn{1}{c}{\begingroup\fontsize{12}{14}\selectfont 17.059\endgroup} & \multicolumn{1}{c}{\begingroup\fontsize{12}{14}\selectfont -44.56\endgroup} & \multicolumn{1}{c}{\begingroup\fontsize{12}{14}\selectfont 275.0\endgroup} & \multicolumn{1}{c}{\begingroup\fontsize{12}{14}\selectfont 69.0000\endgroup} & \multicolumn{1}{c}{\begingroup\fontsize{12}{14}\selectfont 2015\endgroup} & \multicolumn{1}{c}{\begingroup\fontsize{12}{14}\selectfont 152.0\endgroup} & \multicolumn{1}{c}{\begingroup\fontsize{12}{14}\selectfont 7.750\endgroup} & \multicolumn{1}{c}{\begingroup\fontsize{12}{14}\selectfont 7.670\endgroup} & \multicolumn{1}{c}{\begingroup\fontsize{12}{14}\selectfont 7.580\endgroup}\\
\multicolumn{1}{c}{\begingroup\fontsize{12}{14}\selectfont Max.\endgroup} & \multicolumn{1}{c}{\begingroup\fontsize{12}{14}\selectfont 37.090\endgroup} & \multicolumn{1}{c}{\begingroup\fontsize{12}{14}\selectfont 126.09\endgroup} & \multicolumn{1}{c}{\begingroup\fontsize{12}{14}\selectfont 600.0\endgroup} & \multicolumn{1}{c}{\begingroup\fontsize{12}{14}\selectfont 80.0000\endgroup} & \multicolumn{1}{c}{\begingroup\fontsize{12}{14}\selectfont 2018\endgroup} & \multicolumn{1}{c}{\begingroup\fontsize{12}{14}\selectfont 214.0\endgroup} & \multicolumn{1}{c}{\begingroup\fontsize{12}{14}\selectfont 8.750\endgroup} & \multicolumn{1}{c}{\begingroup\fontsize{12}{14}\selectfont 8.670\endgroup} & \multicolumn{1}{c}{\begingroup\fontsize{12}{14}\selectfont 8.500\endgroup}\\
\multicolumn{1}{c}{\begingroup\fontsize{12}{14}\selectfont Mean\endgroup} & \multicolumn{1}{c}{\begingroup\fontsize{12}{14}\selectfont 9.174\endgroup} & \multicolumn{1}{c}{\begingroup\fontsize{12}{14}\selectfont -44.41\endgroup} & \multicolumn{1}{c}{\begingroup\fontsize{12}{14}\selectfont 159.2\endgroup} & \multicolumn{1}{c}{\begingroup\fontsize{12}{14}\selectfont 36.1845\endgroup} & \multicolumn{1}{c}{\begingroup\fontsize{12}{14}\selectfont 2014\endgroup} & \multicolumn{1}{c}{\begingroup\fontsize{12}{14}\selectfont 133.3\endgroup} & \multicolumn{1}{c}{\begingroup\fontsize{12}{14}\selectfont 7.561\endgroup} & \multicolumn{1}{c}{\begingroup\fontsize{12}{14}\selectfont 7.505\endgroup} & \multicolumn{1}{c}{\begingroup\fontsize{12}{14}\selectfont 7.376\endgroup}\\
\multicolumn{1}{c}{\begingroup\fontsize{12}{14}\selectfont Median\endgroup} & \multicolumn{1}{c}{\begingroup\fontsize{12}{14}\selectfont 14.518\endgroup} & \multicolumn{1}{c}{\begingroup\fontsize{12}{14}\selectfont -84.04\endgroup} & \multicolumn{1}{c}{\begingroup\fontsize{12}{14}\selectfont 200.0\endgroup} & \multicolumn{1}{c}{\begingroup\fontsize{12}{14}\selectfont 60.0000\endgroup} & \multicolumn{1}{c}{\begingroup\fontsize{12}{14}\selectfont 2014\endgroup} & \multicolumn{1}{c}{\begingroup\fontsize{12}{14}\selectfont 151.0\endgroup} & \multicolumn{1}{c}{\begingroup\fontsize{12}{14}\selectfont 7.580\endgroup} & \multicolumn{1}{c}{\begingroup\fontsize{12}{14}\selectfont 7.500\endgroup} & \multicolumn{1}{c}{\begingroup\fontsize{12}{14}\selectfont 7.420\endgroup}\\
\addlinespace
\multicolumn{1}{c}{\begingroup\fontsize{12}{14}\selectfont Min.\endgroup} & \multicolumn{1}{c}{\begingroup\fontsize{12}{14}\selectfont -23.563\endgroup} & \multicolumn{1}{c}{\begingroup\fontsize{12}{14}\selectfont -118.28\endgroup} & \multicolumn{1}{c}{\begingroup\fontsize{12}{14}\selectfont 1.0\endgroup} & \multicolumn{1}{c}{\begingroup\fontsize{12}{14}\selectfont 0.4536\endgroup} & \multicolumn{1}{c}{\begingroup\fontsize{12}{14}\selectfont 2011\endgroup} & \multicolumn{1}{c}{\begingroup\fontsize{12}{14}\selectfont 31.0\endgroup} & \multicolumn{1}{c}{\begingroup\fontsize{12}{14}\selectfont 5.080\endgroup} & \multicolumn{1}{c}{\begingroup\fontsize{12}{14}\selectfont 6.170\endgroup} & \multicolumn{1}{c}{\begingroup\fontsize{12}{14}\selectfont 6.170\endgroup}\\
\bottomrule
\end{tabular}


\begin{tabular}{llllllllll}
\toprule
\multicolumn{1}{c}{\begingroup\fontsize{12}{14}\selectfont Statistic\endgroup} & \multicolumn{1}{c}{\begingroup\fontsize{12}{14}\selectfont Acidity\endgroup} & \multicolumn{1}{c}{\begingroup\fontsize{12}{14}\selectfont Body\endgroup} & \multicolumn{1}{c}{\begingroup\fontsize{12}{14}\selectfont Balance\endgroup} & \multicolumn{1}{c}{\begingroup\fontsize{12}{14}\selectfont Uniformity\endgroup} & \multicolumn{1}{c}{\begingroup\fontsize{12}{14}\selectfont Clean.Cup\endgroup} & \multicolumn{1}{c}{\begingroup\fontsize{12}{14}\selectfont Sweetness\endgroup} & \multicolumn{1}{c}{\begingroup\fontsize{12}{14}\selectfont Cupper.Points\endgroup} & \multicolumn{1}{c}{\begingroup\fontsize{12}{14}\selectfont Total.Cup.Points\endgroup} & \multicolumn{1}{c}{\begingroup\fontsize{12}{14}\selectfont Moisture\endgroup}\\
\midrule
\multicolumn{1}{c}{\begingroup\fontsize{12}{14}\selectfont 1st Qu.\endgroup} & \multicolumn{1}{c}{\begingroup\fontsize{12}{14}\selectfont 7.330\endgroup} & \multicolumn{1}{c}{\begingroup\fontsize{12}{14}\selectfont 7.330\endgroup} & \multicolumn{1}{c}{\begingroup\fontsize{12}{14}\selectfont 7.330\endgroup} & \multicolumn{1}{c}{\begingroup\fontsize{12}{14}\selectfont 10.00\endgroup} & \multicolumn{1}{c}{\begingroup\fontsize{12}{14}\selectfont 10.000\endgroup} & \multicolumn{1}{c}{\begingroup\fontsize{12}{14}\selectfont 10.000\endgroup} & \multicolumn{1}{c}{\begingroup\fontsize{12}{14}\selectfont 7.250\endgroup} & \multicolumn{1}{c}{\begingroup\fontsize{12}{14}\selectfont 81.17\endgroup} & \multicolumn{1}{c}{\begingroup\fontsize{12}{14}\selectfont 0.10000\endgroup}\\
\multicolumn{1}{c}{\begingroup\fontsize{12}{14}\selectfont 3rd Qu.\endgroup} & \multicolumn{1}{c}{\begingroup\fontsize{12}{14}\selectfont 7.670\endgroup} & \multicolumn{1}{c}{\begingroup\fontsize{12}{14}\selectfont 7.670\endgroup} & \multicolumn{1}{c}{\begingroup\fontsize{12}{14}\selectfont 7.670\endgroup} & \multicolumn{1}{c}{\begingroup\fontsize{12}{14}\selectfont 10.00\endgroup} & \multicolumn{1}{c}{\begingroup\fontsize{12}{14}\selectfont 10.000\endgroup} & \multicolumn{1}{c}{\begingroup\fontsize{12}{14}\selectfont 10.000\endgroup} & \multicolumn{1}{c}{\begingroup\fontsize{12}{14}\selectfont 7.670\endgroup} & \multicolumn{1}{c}{\begingroup\fontsize{12}{14}\selectfont 83.50\endgroup} & \multicolumn{1}{c}{\begingroup\fontsize{12}{14}\selectfont 0.12000\endgroup}\\
\multicolumn{1}{c}{\begingroup\fontsize{12}{14}\selectfont Max.\endgroup} & \multicolumn{1}{c}{\begingroup\fontsize{12}{14}\selectfont 8.580\endgroup} & \multicolumn{1}{c}{\begingroup\fontsize{12}{14}\selectfont 8.420\endgroup} & \multicolumn{1}{c}{\begingroup\fontsize{12}{14}\selectfont 8.580\endgroup} & \multicolumn{1}{c}{\begingroup\fontsize{12}{14}\selectfont 10.00\endgroup} & \multicolumn{1}{c}{\begingroup\fontsize{12}{14}\selectfont 10.000\endgroup} & \multicolumn{1}{c}{\begingroup\fontsize{12}{14}\selectfont 10.000\endgroup} & \multicolumn{1}{c}{\begingroup\fontsize{12}{14}\selectfont 8.580\endgroup} & \multicolumn{1}{c}{\begingroup\fontsize{12}{14}\selectfont 89.92\endgroup} & \multicolumn{1}{c}{\begingroup\fontsize{12}{14}\selectfont 0.17000\endgroup}\\
\multicolumn{1}{c}{\begingroup\fontsize{12}{14}\selectfont Mean\endgroup} & \multicolumn{1}{c}{\begingroup\fontsize{12}{14}\selectfont 7.517\endgroup} & \multicolumn{1}{c}{\begingroup\fontsize{12}{14}\selectfont 7.494\endgroup} & \multicolumn{1}{c}{\begingroup\fontsize{12}{14}\selectfont 7.492\endgroup} & \multicolumn{1}{c}{\begingroup\fontsize{12}{14}\selectfont 9.87\endgroup} & \multicolumn{1}{c}{\begingroup\fontsize{12}{14}\selectfont 9.846\endgroup} & \multicolumn{1}{c}{\begingroup\fontsize{12}{14}\selectfont 9.933\endgroup} & \multicolumn{1}{c}{\begingroup\fontsize{12}{14}\selectfont 7.461\endgroup} & \multicolumn{1}{c}{\begingroup\fontsize{12}{14}\selectfont 82.05\endgroup} & \multicolumn{1}{c}{\begingroup\fontsize{12}{14}\selectfont 0.09807\endgroup}\\
\multicolumn{1}{c}{\begingroup\fontsize{12}{14}\selectfont Median\endgroup} & \multicolumn{1}{c}{\begingroup\fontsize{12}{14}\selectfont 7.500\endgroup} & \multicolumn{1}{c}{\begingroup\fontsize{12}{14}\selectfont 7.500\endgroup} & \multicolumn{1}{c}{\begingroup\fontsize{12}{14}\selectfont 7.500\endgroup} & \multicolumn{1}{c}{\begingroup\fontsize{12}{14}\selectfont 10.00\endgroup} & \multicolumn{1}{c}{\begingroup\fontsize{12}{14}\selectfont 10.000\endgroup} & \multicolumn{1}{c}{\begingroup\fontsize{12}{14}\selectfont 10.000\endgroup} & \multicolumn{1}{c}{\begingroup\fontsize{12}{14}\selectfont 7.500\endgroup} & \multicolumn{1}{c}{\begingroup\fontsize{12}{14}\selectfont 82.42\endgroup} & \multicolumn{1}{c}{\begingroup\fontsize{12}{14}\selectfont 0.11000\endgroup}\\
\addlinespace
\multicolumn{1}{c}{\begingroup\fontsize{12}{14}\selectfont Min.\endgroup} & \multicolumn{1}{c}{\begingroup\fontsize{12}{14}\selectfont 5.250\endgroup} & \multicolumn{1}{c}{\begingroup\fontsize{12}{14}\selectfont 6.330\endgroup} & \multicolumn{1}{c}{\begingroup\fontsize{12}{14}\selectfont 6.080\endgroup} & \multicolumn{1}{c}{\begingroup\fontsize{12}{14}\selectfont 6.00\endgroup} & \multicolumn{1}{c}{\begingroup\fontsize{12}{14}\selectfont 0.000\endgroup} & \multicolumn{1}{c}{\begingroup\fontsize{12}{14}\selectfont 1.330\endgroup} & \multicolumn{1}{c}{\begingroup\fontsize{12}{14}\selectfont 5.170\endgroup} & \multicolumn{1}{c}{\begingroup\fontsize{12}{14}\selectfont 59.83\endgroup} & \multicolumn{1}{c}{\begingroup\fontsize{12}{14}\selectfont 0.00000\endgroup}\\
\bottomrule
\end{tabular}


\begin{tabular}{llllll}
\toprule
\multicolumn{1}{c}{\begingroup\fontsize{12}{14}\selectfont Statistic\endgroup} & \multicolumn{1}{c}{\begingroup\fontsize{12}{14}\selectfont Category.One.Defects\endgroup} & \multicolumn{1}{c}{\begingroup\fontsize{12}{14}\selectfont Quakers\endgroup} & \multicolumn{1}{c}{\begingroup\fontsize{12}{14}\selectfont Category.Two.Defects\endgroup} & \multicolumn{1}{c}{\begingroup\fontsize{12}{14}\selectfont Altitude.Mean.Meters\endgroup} & \multicolumn{1}{c}{\begingroup\fontsize{12}{14}\selectfont Weight\_Bags\endgroup}\\
\midrule
\multicolumn{1}{c}{\begingroup\fontsize{12}{14}\selectfont 1st Qu.\endgroup} & \multicolumn{1}{c}{\begingroup\fontsize{12}{14}\selectfont 0.0000\endgroup} & \multicolumn{1}{c}{\begingroup\fontsize{12}{14}\selectfont 0.0000\endgroup} & \multicolumn{1}{c}{\begingroup\fontsize{12}{14}\selectfont 0.000\endgroup} & \multicolumn{1}{c}{\begingroup\fontsize{12}{14}\selectfont 1100\endgroup} & \multicolumn{1}{c}{\begingroup\fontsize{12}{14}\selectfont 180\endgroup}\\
\multicolumn{1}{c}{\begingroup\fontsize{12}{14}\selectfont 3rd Qu.\endgroup} & \multicolumn{1}{c}{\begingroup\fontsize{12}{14}\selectfont 0.0000\endgroup} & \multicolumn{1}{c}{\begingroup\fontsize{12}{14}\selectfont 0.0000\endgroup} & \multicolumn{1}{c}{\begingroup\fontsize{12}{14}\selectfont 5.000\endgroup} & \multicolumn{1}{c}{\begingroup\fontsize{12}{14}\selectfont 1550\endgroup} & \multicolumn{1}{c}{\begingroup\fontsize{12}{14}\selectfont 17250\endgroup}\\
\multicolumn{1}{c}{\begingroup\fontsize{12}{14}\selectfont Max.\endgroup} & \multicolumn{1}{c}{\begingroup\fontsize{12}{14}\selectfont 31.0000\endgroup} & \multicolumn{1}{c}{\begingroup\fontsize{12}{14}\selectfont 11.0000\endgroup} & \multicolumn{1}{c}{\begingroup\fontsize{12}{14}\selectfont 47.000\endgroup} & \multicolumn{1}{c}{\begingroup\fontsize{12}{14}\selectfont 2560\endgroup} & \multicolumn{1}{c}{\begingroup\fontsize{12}{14}\selectfont 37950\endgroup}\\
\multicolumn{1}{c}{\begingroup\fontsize{12}{14}\selectfont Mean\endgroup} & \multicolumn{1}{c}{\begingroup\fontsize{12}{14}\selectfont 0.4263\endgroup} & \multicolumn{1}{c}{\begingroup\fontsize{12}{14}\selectfont 0.1463\endgroup} & \multicolumn{1}{c}{\begingroup\fontsize{12}{14}\selectfont 3.822\endgroup} & \multicolumn{1}{c}{\begingroup\fontsize{12}{14}\selectfont 1293\endgroup} & \multicolumn{1}{c}{\begingroup\fontsize{12}{14}\selectfont 6558\endgroup}\\
\multicolumn{1}{c}{\begingroup\fontsize{12}{14}\selectfont Median\endgroup} & \multicolumn{1}{c}{\begingroup\fontsize{12}{14}\selectfont 0.0000\endgroup} & \multicolumn{1}{c}{\begingroup\fontsize{12}{14}\selectfont 0.0000\endgroup} & \multicolumn{1}{c}{\begingroup\fontsize{12}{14}\selectfont 2.000\endgroup} & \multicolumn{1}{c}{\begingroup\fontsize{12}{14}\selectfont 1311\endgroup} & \multicolumn{1}{c}{\begingroup\fontsize{12}{14}\selectfont 640\endgroup}\\
\addlinespace
\multicolumn{1}{c}{\begingroup\fontsize{12}{14}\selectfont Min.\endgroup} & \multicolumn{1}{c}{\begingroup\fontsize{12}{14}\selectfont 0.0000\endgroup} & \multicolumn{1}{c}{\begingroup\fontsize{12}{14}\selectfont 0.0000\endgroup} & \multicolumn{1}{c}{\begingroup\fontsize{12}{14}\selectfont 0.000\endgroup} & \multicolumn{1}{c}{\begingroup\fontsize{12}{14}\selectfont 1\endgroup} & \multicolumn{1}{c}{\begingroup\fontsize{12}{14}\selectfont 1\endgroup}\\
\bottomrule
\end{tabular}
\end{center} 
\end{landscape}
}

\newpage

\begingroup
\setlength{\parindent}{-0.5in}
\setlength{\leftskip}{0.5in}

\hypertarget{refs}{}

\endgroup

\printbibliography

\end{document}
